Isabelle is a logic of total functions. Nonetheless, we can define
apparently partial functions, such as \verb|hd|:
\begin{verbatim}
fun hd :: "'a list => 'a" where
  "hd (x#xs) = x"
\end{verbatim}

How do we reconcile \verb|hd| being partial with Isabelle functions
being total? The answer is that in Isabelle, \verb|hd| is total, but
the behaviour of \verb|hd []| is unspecified: it returns some
arbitrary value of type \verb|'a|. Meanwhile in Haskell, \verb|head|
is partial, but the behaviour of \verb|head []| is specified: it
crashes.

Isabelle and Haskell give different semantics to partial functions. We
must respect this difference when translating Isabelle to Haskell,
otherwise HipSpec might find equations which are not true in Isabelle
or miss equations which are not true in Haskell.\footnote{As all our
lemmas are proved in Isabelle, at least we do not risk unsoundness.}
For example, EXAMPLE.
