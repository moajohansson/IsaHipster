\section{Introduction}
\label{sec:intro}
The concept of theory exploration was first introduced by Buchberger \cite{buchberger2000theory}. He argues that as a contrast to  automated theorem provers that focus on proving one theorem at the time in isolation, mathematicians instead typically proceed by exploring entire theories, by conjecturing and proving layers of increasingly complex propositions. For each layer, appropriate proof methods are identified, and previously proved lemmas may be used to prove later conjectures. When a new concept (e.g. a new function) is introduced, we should prove a set of new conjectures which, ideally, ``completely" relates the new with the old, after which other propositions in this layer can be proved easily by easy ``routine" reasoning. Mathematical software should be designed to support this workflow. This is arguably the mode of use supported by many interactive proof assistants, such as Theorema \cite{theorema} and Isabelle \cite{isabelle}. However, they leave the generation of new conjectures relating different concepts largely to the user. Recently, a number of different systems have been implemented to address also this aspect of theory exploration \cite{McCasland2006,isacosy,isascheme,hipspecCADE}.  In this work, we take this one step further by integrating the discovery and proof of new conjectures in the workflow of the interactive theorem prover Isabelle. Our system, called Hipster, is based on our previous work on HipSpec \cite{hipspecCADE}, a theory exploration system for Haskell programs. In previous work \cite{hipspecCADE}, we showed that HipSpec is able to automatically discover many of the equational theorems present in Isabelle's list library. In this article we show how this can be used to speed up the development of new theories in Isabelle by discovering basic lemmas automatically. 

Hipster translate Isabelle theories into Haskell, after which computation and evaluation in Haskell is used to efficiently generate generate interesting conjectures. These are then imported back into Isabelle and proved by automated tactics. Hipster can be used in two main ways: in \emph{exploratory mode} it quickly discovers basic properties about a newly defined function and its relationship to already existing ones. Hipster can also be used in \emph{proof mode}, to provide lemma hints for an ongoing proof attempt when the user is stuck. 

Our work also fits nicely with the increased popularity of Sledgehammer \cite{sledgehammer}, an Isabelle tool allowing the user to call various external automated provers. Sledgehammer use \emph{relevance filtering} to select among the available lemmas those likely to be useful for proving a given conjecture \cite{mash}. However, if a crucial lemma is missing, the proof attempt will fail. If theory exploration is employed before, we can increase the chances of Sledgehammer succeeding with little user effort. 

HERE: short example? Something unprovable by Sledgehammer before theory exploration.

%Hipster is parametrised by tactics for routine and "hard" reasoning. In this paper, we consider routine reasoning to be first-order resoning by Metis, give the conjectures Hipster has proved so far while "hard" reasoning is those proofs requiring induction. 

%Sledgehammer is a popular and powerful tool in Isabelle which allows users to call various external automated first-order provers \cite{sledgehammer}. Proofs are then replayed inside Isabelle's trusted kernel using its Metis prover. Sledgehammer's machine learning based relevance filter \cite{mash}, selects a number of facts from Isabelle's library that are judged to likely be relevant for the problem at hand. 
% 
%Which is because we can improve on Sledgehammer-style interactions in cases when the required lemmas simply are not available, so Sledgehammer would fail. 
Integrating the theory exploration system in an interactive setting has several other advantages, for instance when exploring a large theory which otherwise would generate a very large search space. The user can instead incrementally explore different relevant sub-theories while avoiding a search space explosion. Lemmas discovered in each sub-theory can be made available when exploring increasingly larger sets of functions. 

The article is organised as follows: In section \ref{sec:background} we give a brief overview of the HipSpec system which Hipster use to generate conjectures, after which we describe Hipster in more detail in section \ref{sec:hipster}, together with some worked examples of how it can be used. In section \ref{sec:partial} we describe how we deal with partial functions, as Haskell and Isabelle differ in their semantics for these. 
