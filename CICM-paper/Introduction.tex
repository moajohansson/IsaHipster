\section{Introduction}
\label{sec:intro}
The concept of theory exploration was first introduced by Buchberger \cite{buchberger2000theory}. He argues that as a contrast to  automated theorem provers that focus on proving one theorem at the time in isolation, mathematicians instead typically proceed by exploring entire theories, by conjecturing and proving layers of increasingly complex propositions. For each layer, appropriate proof methods are identified, and previously proved lemmas may be used to prove later conjectures. When a new concept (e.g. a new function) is introduced, we should prove a set of new conjectures which, ideally, "completely" relates the new with the old, after which other propositions in this layer can be proved easily by easy "routine" reasoning. Mathematical software should be designed to support this workflow. This is arguably what many interactive proof assistants, such as Theorema \cite{theorema} and Isabelle \cite{isabelle}, roughly support. However, they leave the generation of new conjectures relating different concepts largely to the user. Recently, a number of different systems have been implemented to address also this aspect of theory exploration \cite{McCasland2006,isacosy,isascheme,hipspecCADE}.  In this work, we take this one step further by integrating the discovery and proof of new conjectures in the workflow of the interactive theorem prover Isabelle. Our system, called Hipster, is based on our previous work on HipSpec \cite{hipspecCADE}, a theory exploration for Haskell programs. Isabelle theories are translated into Haskell, after which computation and evaluation in Haskell is used to efficiently generate generate interesting conjectures, which are then imported back into Isabelle and proved by automated tactics. Hipster can be used in two main ways, either in exploratoty mode, to quickly discover basic properties about a newly defined function and its relationship to already existing ones. It can also be used to provide lemma hints for an ongoing proof attempt when the user is stuck. 

%Hipster is parametrised by tactics for routine and "hard" reasoning. In this paper, we consider routine reasoning to be first-order resoning by Metis, give the conjectures Hipster has proved so far while "hard" reasoning is those proofs requiring induction. 

Sledgehammer is a popular and powerful tool in Isabelle which allows users to call various external automated first-order provers \cite{sledgehammer}. Proofs are then replayed inside Isabelle's trusted kernel using its Metis prover. Sledgehammer's machine learning based relevance filter \cite{mash}, selects a number of facts from Isabelle's library that are judged to likely be relevant for the problem at hand. 
 
Which is because we can improve on Sledgehammer-style interactions in cases when the required lemmas simply are not available, so Sledgehammer would fail. 

The Hipster framework improve on the flexibility and usability over previous systems. 
 + Flexilibity and interactivity, user can choose to exlore a large theory incrementally in sub-theories. Fast!
+ Translation between Isabelle and Haskell.  Blanchette variables, as Isabelle and Haskell deal with partial functions and their semantics in different ways.  
+ Reuse 

The article is organised as follows: In section \ref{sec:background} we give a brief overview of the HipSpec system which Hipster use to generate conjectures, after which we describe Hipster in more detail in section \ref{sec:hipster}. Blah bla... 
