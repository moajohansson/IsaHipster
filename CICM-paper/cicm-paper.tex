\documentclass{llncs}
\usepackage{tikz}
\usetikzlibrary{shapes,arrows,calc}


%\usepackage{url}
\usepackage{multirow}
\usepackage{listings}
\usepackage{amsmath}  % for equation*
\usepackage{array}    % for tabular
\usepackage{verbatim} % for comment
\usepackage{wrapfig}
\usepackage[final]{microtype}
\usepackage[pdfborder={0 0 0}]{hyperref}

\newcommand\forAll[1]{\forall \, #1 \, . \,}
\newcommand\forAllII[2]{\forall \, #1 \, #2 \, . \,}

\newcommand\propno[1]{(\emph{#1})}
\newcommand\hs[1]{\texttt{#1}}

% \raggedbottom

\lstnewenvironment{code}[1][]
  {\noindent
   \vspace{-0.5\baselineskip}
   \lstset{basicstyle=\ttfamily,
           frame=single,
           language=Haskell,
           keywordstyle=\color{black},
           #1}
   \fontsize{8pt}{8pt}\selectfont}
  {}

\newcommand\NOTE[1]{} % \mbox{}\marginpar{\fontsize{8pt}{8pt}\selectfont\raggedright\hspace{0pt}\emph{#1}}}

\def\maindocument{} % To tell tikz images that they are not stand alone

\begin{document}

\title{Integrating Theory Exploration in a Proof Assistant}

\author{Moa Johansson \and Dan Ros\'en \and Nicholas Smallbone}
\institute{Department of Computer Science and Engineering, Chalmers University of Technology
	\email{\{moa.johansson,nicsma,danr\}@chalmers.se}
	}

\authorrunning{Johansson, Ros\'en and Smallbone}
\titlerunning{Integrating Theory Exploration in a Proof Assistant}

\maketitle

\begin{abstract}
Theory exploration is a technique for automatically discovering novel and interesting lemmas in a given mathematical theory development. This paper describes Hipster, a system integrating theory exploration with the proof assistant Isabelle.
Existing Isabelle-tools like Sledgehammer can automatically select existing lemmas from the libraries that are likely to be needed for automation of many proofs. Hipster is an extension of this approach which also allows for discovery of new or missing lemmas which require induction, thus increasing the number of proofs that can be automated.
Hipster can be used in two main ways, either in exploratory mode to automatically generate basic lemmas about a given set of datatypes and functions in a new theory development, or in proof mode, for a particular proof attempt, trying to discover the missing lemmas which would allow the current goal to be proved. 

\end{abstract}

\section{Introduction}
\label{sec:intro}
The concept of theory exploration was first introduced by Buchberger \cite{buchberger2000theory}. He argues that as a contrast to  automated theorem provers that focus on proving one theorem at the time in isolation, mathematicians instead typically proceed by exploring entire theories, by conjecturing and proving layers of increasingly complex propositions. For each layer, appropriate proof methods are identified, and previously proved lemmas may be used to prove later conjectures. When a new concept (e.g. a new function) is introduced, we should prove a set of new conjectures which, ideally, ``completely" relates the new with the old, after which other propositions in this layer can be proved easily by easy ``routine" reasoning. Mathematical software should be designed to support this workflow. This is arguably the mode of use supported by many interactive proof assistants, such as Theorema \cite{theorema} and Isabelle \cite{isabelle}. However, they leave the generation of new conjectures relating different concepts largely to the user. Recently, a number of different systems have been implemented to address also this aspect of theory exploration \cite{McCasland2006,isacosy,isascheme,hipspecCADE}.  In this work, we take this one step further by integrating the discovery and proof of new conjectures in the workflow of the interactive theorem prover Isabelle. Our system, called Hipster, is based on our previous work on HipSpec \cite{hipspecCADE}, a theory exploration system for Haskell programs. In previous work \cite{hipspecCADE}, we showed that HipSpec is able to automatically discover many of the equational theorems present in Isabelle's list library. In this article we show how this can be used to speed up the development of new theories in Isabelle by discovering basic lemmas automatically. 

Hipster translate Isabelle theories into Haskell, after which computation and evaluation in Haskell is used to efficiently generate generate interesting conjectures. These are then imported back into Isabelle and proved by automated tactics. Hipster can be used in two main ways: in \emph{exploratory mode} it quickly discovers basic properties about a newly defined function and its relationship to already existing ones. Hipster can also be used in \emph{proof mode}, to provide lemma hints for an ongoing proof attempt when the user is stuck. 

Our work also fits nicely with the increased popularity of Sledgehammer \cite{sledgehammer}, an Isabelle tool allowing the user to call various external automated provers. Sledgehammer use \emph{relevance filtering} to select among the available lemmas those likely to be useful for proving a given conjecture \cite{mash}. However, if a crucial lemma is missing, the proof attempt will fail. If theory exploration is employed before, we can increase the chances of Sledgehammer succeeding with little user effort. 

HERE: short example? Something unprovable by Sledgehammer before theory exploration.

%Hipster is parametrised by tactics for routine and "hard" reasoning. In this paper, we consider routine reasoning to be first-order resoning by Metis, give the conjectures Hipster has proved so far while "hard" reasoning is those proofs requiring induction. 

%Sledgehammer is a popular and powerful tool in Isabelle which allows users to call various external automated first-order provers \cite{sledgehammer}. Proofs are then replayed inside Isabelle's trusted kernel using its Metis prover. Sledgehammer's machine learning based relevance filter \cite{mash}, selects a number of facts from Isabelle's library that are judged to likely be relevant for the problem at hand. 
% 
%Which is because we can improve on Sledgehammer-style interactions in cases when the required lemmas simply are not available, so Sledgehammer would fail. 
Integrating the theory exploration system in an interactive setting has several other advantages, for instance when exploring a large theory which otherwise would generate a very large search space. The user can instead incrementally explore different relevant sub-theories while avoiding a search space explosion. Lemmas discovered in each sub-theory can be made available when exploring increasingly larger sets of functions. 

The article is organised as follows: In section \ref{sec:background} we give a brief overview of the HipSpec system which Hipster use to generate conjectures, after which we describe Hipster in more detail in section \ref{sec:hipster}, together with some worked examples of how it can be used. In section \ref{sec:partial} we describe how we deal with partial functions, as Haskell and Isabelle differ in their semantics for these. 


\section{Background}
\label{sec:background}

In this section we give a brief overview of the HipSpec system which
we use as a backend for generating conjectures. HipSpec is a
state-of-the-art inductive theorem prover and theory exploration
system for Haskell. In \cite{hipspecCADE} we showed that HipSpec is
able to automatically discover and prove the kind of equational lemmas present in
Isabelle's libraries, when given the corresponding functions written in Haskell.

HipSpec works in two stages:
\begin{enumerate}
\item Generate a set of conjectures about the functions at hand. These conjectures are equations between terms involving the given functions, and are only tested for correctness.

\item Attempt to prove each of the conjectures, using already proven conjectures as assumptions. HipSpec implements this by enumerating induction schemas, and firing off many proof obligations to automated first-order logic theorem provers.
\end{enumerate}
The proving power of HipSpec comes from its capability to
automatically discover and prove lemmas, which are then used to help
subsequent proofs.

In Hipster we are not interested in HipSpec's
proving capabilities (stage (2) above); we use Isabelle for the proofs instead. Isabelle
is based on a small core of trusted axioms, and proofs must be built
on top of those axioms. In other words, we would have to reconstruct
inside Isabelle any proof we found anyway, so it is easier to use
Isabelle for the proofs in the first place.

The part of HipSpec we do use
is its conjecture synthesis system (part (1) above), called QuickSpec \cite{quickspec}),
which efficiently generates equations about given functions and
datatypes.

QuickSpec takes a set of functions as input, and proceeds to generate all
type-correct terms up to a given limit (usually up to depth three), with
variables (usually three per type).  It first assumes that all terms the same
type are equivalent, and initially puts them in the same equivalence class.  By
generating random values for the variables (using QuickCheck \cite{quickcheck}), the terms are repeatedly evaluated and their equivalence
classes are separated according to the evaluation results.  This process is
repeated until the equivalence classes stabilise, and resulting equations can
be read off from each equivalence class.  This means that the conjectures
generated are, although not yet proved, fairly likely to be true.

As an example, we tell QuickSpec to explore the theory with list append,
\verb~@~, the empty list, \verb~[]~, and three list variables \verb~xs~,
\verb~ys~, \verb~zs~. Some of all the terms it will generate are these:
\verb~(xs@ys)@zs~, \verb~xs@(ys@zs)~, \verb~xs@[]~ and simply \verb~xs~.
Initially, all four will be assumed to be in the same equivalence class.
Before soon, it will generate non-empty lists for \verb~ys~ and \verb~zs~.
Evaluating the terms with such assignments will split this equivalence class
into two: one with the former two and one with the latter. After this, more
values will be generated for these lists variables, but no evaluation will make
them split again. Eventually, QuickSpec stops and the equations for
associativity and right identity can be read off.



\section{Hipster: Implementation and Use}
\label{sec:hipster}

In this section we give an overview of Hipster, and show how it can be used both in theory exploration mode and in proof mode, to find lemmas relevant for a particular proof attempt. An overview of Hipster is shown in figure \ref{fig:hipster}. 

\begin{figure}[htbp]
\begin{center}
\includegraphics[scale=0.45]{hipster}

\caption{Overview of Hipster}
\label{fig:hipster}
\end{center}
\end{figure}

Starting from an Isabelle theory, Hipster calls Isabelle's code generator \cite{codegen} to translate the given functions into a Haskell program. In order to use the testing framework from QuickCheck, as described in the previous section, we must however also post-process the Haskell file, adding \emph{generators}, which are responsible for producing arbitrary values used for evaluation and testing. We use generators automatically deduced by the Feat package \cite{feat}. Another important issue that need to be addressed at this stage is the differences in semantics for partial functions in Isabelle and Haskell. In order to avoid HipSpec missing equations that hold in Isabelle, but not in Haskell, we had to modify the translation of partial functions. This is explained in more detail in section \ref{sec:partial}.

Once the Haskell program is in place, we run theory exploration and generate a set of equational conjectures, which HipSpec orders according to generality. More general equations are preferred, as we expect these to be more widely applicable as lemmas. In previous work on HipSpec, the system would at this stage apply induction on the conjectures and send them off to some external prover. Here, we instead import them back into Isabelle as we wish to produce checkable LCF-style proofs for our conjectures. 

The proof procedure in Hipster is parametrised by two tactics, one for easy or \emph{routine reasoning} and one for \emph{difficult reasoning}. In the examples below, we use Isabelle's simplifier followed by first-order reasoning by Metis as \emph{routine reasoning}, and a tactic performing structural induction followed by simplification and first-order reasoning as \emph{difficult reasoning}. Both tactics have access to the theorems proved so far, and hence gets stronger as the proof procedure proceed through the list of conjectures. As there are rather many conjectures produced by theory exploration, we do not want to present them all to the user, but rather select the most interesting ones, which are difficult to prove. Those that follow only by routine reasoning are filtered out. 
Depending on the theory and application we can change these tactics to suit our needs. If we want Hipster to produce fewer or more lemmas, we can choose a stronger or weaker tactic, allowing for flexibility.  

The order in which Hipster tries to prove things matter. As we mentioned, it will try more general conjectures first, with the hope that they will be useful to filter out many more specific routine results. Occasionally though, a proof will fail as a not yet proved lemma is required. In this case, the failed conjecture is added back into the list of open conjectures and retried later, provided that at least one new lemma has been proved in the meantime to ensure progress and termination. Hipster terminates when it either runs out of open conjectures, or when it does not make any more progress. 

\subsection{Exploring a Theory of Binary Trees}
In this example we look at a theory about binary trees, with data stored at the leaves:
\begin{verbatim}
datatype 'a Tree = 
  Leaf 'a 
  | Node 'a Tree  'a Tree
\end{verbatim}

\subsection{Proving Correctness of a Small Compiler}
The following example is about a compiler to a stack machine for a toy language with generic types of expressions\footnote{This example a slight variation of that in \S3.3 in the Isabelle tutorial \cite{isabelle}. It had to be modified as our generators on the Haskell side does not support generation of values for datatype constructors involving higher-order arguments}. We show how theory exploration can be used to unblock a proof on which automated tactics otherwise fail due to a missing lemma.
Expressions in the language are built from constants (\texttt{Cex}), values (\texttt{Vex}) and binary operators (\texttt{Bex}): 
\begin{verbatim}
datatype ('c, 'v, 'b) expr =
  Cex 'c |
  Vex 'v |
  Bex "'b" "('c,'v,'b) expr" "('c,'v,'b) expr"
\end{verbatim}
The types of variables, values and binary operators are not fixed, but given by type parameters \texttt{'c}, \texttt{'v} and \texttt{'b}. 
To evaluate an expression, we define a function \texttt{value}, parametrised by a lookup function for binary operators and an environment mapping variables to values:
\begin{verbatim}
fun value::('b =>'c =>'c =>'c) => ('v =>'c) => ('c,'v,'b) expr =>'c
where
   value getBinop env (Cex c) = c
 | value getBinop env (Vex v) = env v
 | value getBinop env (Bex b e1 e2) = 
    (getBinop b) (value getBinop env e1) (value getBinop env e2)
\end{verbatim}
A program for our stack machine consists of four instructions:
\begin{verbatim}
datatype ('c, 'v, 'b) program =
    Done
  | Const 'c "('c, 'v, 'b) program
  | Load 'v  "('c, 'v, 'b) program
  | Apply 'b "('c, 'v, 'b) program
\end{verbatim}
A program is either empty (\texttt{Done}), or consists of one of the instructions \texttt{Const}, \texttt{Load} or \texttt{Apply}, followed by the remaining program. We further define a function \texttt{sequence} for combining programs:
\begin{verbatim}
fun sequence::('c,'v,'b) program => ('c,'v,'b) program => ('c,'v,'b) program
where
    sequence Done p = p
  | sequence (Const c p) p' = Const c (sequence p p')
  | sequence (Load v p) p' = Load v (sequence p p')
  | sequence (Apply b p) p' = Apply b (sequence p p')
\end{verbatim}
Program execution is modelled by the function \texttt{exec}, which given a lookup function for binary operators, a store for variables and a program, returns the values on the stack after execution.
\begin{verbatim}
fun exec::('b =>'c =>'c =>'c)=>('v =>'c)=>('c,'v,'b) program =>'c list =>'c list
where
    exec getBinop env Done stack = stack
  | exec getBinop env (Const c p) stack = exec getBinop env p (c#stack) 
  | exec getBinop env (Load v p) stack = exec getBinop env p ((env v)#stack)
  | exec getBinop env (Apply b p) stack = exec getBinop env p 
  	  ((getBinop b (hd stack) (hd(tl stack)))#(tl(tl stack)))
\end{verbatim}
We finally define a function \texttt{compile}, which specifies how expressions are compiled into programs:
\begin{verbatim}
fun compile::('c,'v,'b) expr => ('c,'v,'b) program
  where
    compile (Cex c) =  Const c Done
  | compile (Vex v) =  Load v Done
  | compile (Bex b e1 e2) = 
     sequence (compile e2) (sequence (compile e1) (Apply b Done))
\end{verbatim}
Now, we wish to prove correctness of the compiler, namely that executing a compiled expression indeed results in the value of that expression: 
\begin{verbatim}
theorem "exec getBinop env (compile e) [] = [value getBinop env e]"
\end{verbatim}
If we try to apply induction on \texttt{e}, Isabelle's simplifier solves the base-case but neither the simplifier nor Sledgehammer succeeds in proving the step-case. At this stage, we can apply Hipster's theory exploration tactic. It will generate a set of conjectures, and interleave proving these with trying to prove the open sub-goal. Once it succeeds, it presents the user with a list of lemmas it has proved, in this case:
\begin{small}
\begin{verbatim}
lemma lemma_a: "sequence x Done = x"
by (tactic {* Hipster_Tacs.induct_simp_metis . . . *})

lemma lemma_aa: "exec x y (sequence z x1) xs = exec x y x1 (exec x y z xs)"
by (tactic {* Hipster_Tacs.induct_simp_metis . . . *})

lemma lemma_ab: "exec x y (compile z) xs = value x y z # xs"
by (tactic {* Hipster_Tacs.induct_simp_metis . . . *})
\end{verbatim}
\end{small}
Pasting these into our proof script we can try Sledgehammer on our theorem again. This time, it succeeds and suggests the one line proof: \texttt{by (metis lemma\_ab)}.
%\begin{verbatim}
%theorem "exec getBinop env (compile e) [] = [value getBinop env e]"
%by (metis lemma_ab)
%\end{verbatim}



% Dealing with Partial Functions
\section{Dealing With Partial Functions}
\label{sec:partial}
Isabelle is a logic of total functions. Nonetheless, we can define
apparently partial functions, such as \verb|hd|:
\begin{verbatim}
fun hd :: 'a list => 'a where
  hd (x#xs) = x
\end{verbatim}

How do we reconcile \verb|hd| being partial with Isabelle functions
being total? The answer is that in Isabelle, \verb|hd| is total, but
the behaviour of \verb|hd []| is unspecified: it returns some
arbitrary value of type \verb|'a|. Meanwhile in Haskell, \verb|head|
is partial, but the behaviour of \verb|head []| is specified: it
crashes. We must therefore translate \emph{partially defined} Isabelle
functions into \emph{total but underspecified} Haskell functions.

Hipster uses a technique suggested by Jasmin Blanchette
\cite{blanchettification} to deal with partial functions. Whenever we translate an Isabelle function
that is missing some cases, we add a default case, like so:
\begin{verbatim}
hd :: [a] -> a
hd (x:xs) = x
hd [] = ???
\end{verbatim}

But what should we put for the result of \verb|hd []|? To model the
notion that \verb|hd []| is unspecified, whenever we evaluate a test
case we will pick a \emph{random} value for \verb|hd []|. This value
will vary from test case to test case but will be consistent within
one run of a test case. The idea is that, if an equation involving
\verb|hd| in Haskell no matter which value we pick for \verb|hd []|,
it will also hold in Isabelle where the value of \verb|hd []| is unspecified.

Suppose we define the function \verb|second|, which returns the second
element of a list, as
\begin{verbatim}
second (x#y#xs) = y
\end{verbatim}
It might seem that we should translate \verb|second|, by analogy with \verb|hd|, as
\begin{verbatim}
second :: [a] -> a
second (x:y:xs) = y
second _ = ???
\end{verbatim}
and pick a random value of type \verb|a| to use in the default case.
But this translation is wrong! If we apply our translated \verb|second|
to a single-element list, it will give the same answer regardless of which
element is in the list, and HipSpec will discover the lemma
\verb|second [x] = second [y]|. This lemma is certainly not true of our
Isabelle function, which says nothing about the behaviour
of \verb|second| on single-element lists, and Hipster will fail to
prove it.

We must allow the default case to produce a different result for
different arguments. We therefore translate \verb|second| as
\begin{verbatim}
second :: [a] -> a
second (x:y:xs) = y
second xs = ??? xs
\end{verbatim}
where \verb|???| is a random \emph{function} of type \verb|[a] -> a|.
(QuickCheck can generate random functions.) As before, whenever we
evaluate a test case, we instantiate \verb|???| with a new random
function\footnote{To avoid having to retranslate the Isabelle theory
every time we evaluate a test case, in reality we parametrise the
generated program on the various \texttt{???} functions. That way,
whenever we evaluate a test case, we can cheaply change the default cases.}.
This second translation mimics Isabelle's semantics: any equation that
holds in Haskell no matter how we instantiate the \verb|???| functions
also holds in Isabelle.


\section{Related Work}
%\section{Related Work}
\label{sec:related}

Hipster is an extension to our previous work on the HipSpec system \cite{hipspecCADE}, which was not designed for use in an interactive setting. HipSpec applies structural induction to conjectures generated by QuickSpec and sends off proof obligations to external first-order provers. Hipster short-circuits the proof part and directly imports the conjectures back into Isabelle. This allows for more flexibility in the choice of tactics employed, by letting the user control what is to be considered routine and difficult reasoning. Being inside Isabelle/HOL provides the possibility to easily record lemmas for future use, perhaps in other theory developments. It gives us the possibility to re-check proofs if required, as well as increased reliability as proofs have been run through Isabelle's trusted kernel. As Hipster uses HipSpec for conjecture generation, any difference in performance (e.g. speed, lemmas proved) will depend only on what prover backend is used by HipSpec and what tactic is used by Hipster. %We remark that some tactics, such as Isabelle's simplifier, can sometimes even make Hipster faster than HipSpec despite running the proofs through Isabelle's , but this depend very much on the theory

There are two other theory exploration systems available for Isabelle/HOL, IsaCoSy \cite{isacosy} and IsaScheme \cite{isascheme}. They differ in the way they generate conjectures, and both discover similar lemmas as HipSpec/Hipster. A comparison between HipSpec, IsaCoSy and IsaScheme can be found in \cite{hipspecCADE}, showing that all three systems manage to find largely the same lemmas on theories about lists and natural numbers. HipSpec does however outperform the other two systems on speed.
IsaCoSy ensures that terms generated are non-trivial to prove by only generating irreducible terms, i.e. conjectures that do not have simple proofs by equational reasoning. These are then filtered through a counter-example checker and passed to IsaPlanner for proof \cite{isaplanner}. IsaScheme, as the name suggests, follows the scheme-based approach first introduced for algorithm synthesis in Theorema \cite{theorema}. IsaScheme uses general user-specified schemas describing the shape of conjectures and then instantiates them with available functions and constants. It combines this with counter-example checking and Knuth-Bendix completion techniques in an attempt to produce a minimal set of lemmas. 

Unfortunately, the counter-example checking in IsaCoSy and IsaScheme
is often too slow for use in an interactive setting. Unlike IsaCoSy,
Hipster may generate reducible terms, but thanks to the equivalence
class reasoning in QuickSpec, testing is much more efficient, and
conjectures with trivial proofs are instead quickly filtered out at
the proof stage. None of our examples takes more than twenty seconds
to run.

Neither IsaCoSy or IsaScheme has been used to generate lemmas in stuck proof attempts, but only in fully automated exploratory mode.
Starting from stuck proof attempts allows us to reduce the size of the
interesting background theory, which promises better scalability.

\emph{Proof planning critics} have been employed to analyse failed
proof attempts in automatic inductive proofs \cite{productiveuse}. The
critics use information from the failure in order to try to speculate
a missing lemma top-down, using techniques based on rippling and
higher-order unification. Hipster (and HipSpec) takes a less
restricted approach, instead constructing lemmas bottom-up, from the symbols available. As was shown in our previous work \cite{hipspecCADE}, this succeeds in finding lemmas that the top-down critics based approach fails to find, at the cost of possibly also finding a few extra lemmas as we saw in the example in section \ref{sec:comp-ex}.

%Hipster has instead been designed specifically to be used interactively during theory development, allowing for more flexibility and user control, both in exploratory and proof mode. For instance, the user can specify what is to be considered difficult or routine reasoning, thus affecting the outcome of the exploration.  

%MATHsAiD, HR?
%Other systems not integrated in the workflow of a proof-assistant. With all the advantages that entail, recording discovered lemmas etc. Building up theories incrementally.

%Unlike HipSpec User can control search space by selecting which functions to pass in together. May run several passes of exploration with different combinations of functions. Parametrised by routine/difficult tactics for flexibility.


Other systems not integrated in the workflow of a proof-assistant. 


\section{Conclusion and Further Work}
%\section{Conclusion and Further Work}
\label{sec:concl}

Hipster integrates lemma discovery by theory exploration in the proof assistant Isabelle. We demonstrated two typical use cases of how this can help and speed up theory development: by generating interesting basic lemmas in \emph{exploration mode} or as a lemma suggestion mechanism for a stuck proof attempt in \emph{proof mode}. Hipster thus becomes a complement to tools like Sledgehammer, by discovery of missing lemmas, more proofs can be tackled automatically. 

The lemma discovery is currently limited to equational lemmas. We plan to extend the term generation and evaluation on the Haskell side to also take side conditions into account. For example, if theory exploration is called in the middle of a proof attempt, there may be assumptions associated with the current sub-goal, which should be taken into account when evaluating terms and dividing them into equivalence classes. 


%Conditionals, analogy between different datastructures when similar lemmas has been found about them. 

Conditionals, analogy between different datastructures when similar lemmas has been found about them. 

\bibliographystyle{plain}
\bibliography{bibfile}

\end{document}
