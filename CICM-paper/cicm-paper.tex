\documentclass{llncs}
\usepackage{tikz}
\usetikzlibrary{shapes,arrows,calc}


%\usepackage{url}
\usepackage{multirow}
\usepackage{listings}
\usepackage{amsmath}  % for equation*
\usepackage{array}    % for tabular
\usepackage{verbatim} % for comment
\usepackage{wrapfig}
\usepackage[final]{microtype}
\usepackage[pdfborder={0 0 0}]{hyperref}

\newcommand\forAll[1]{\forall \, #1 \, . \,}
\newcommand\forAllII[2]{\forall \, #1 \, #2 \, . \,}

\newcommand\propno[1]{(\emph{#1})}
\newcommand\hs[1]{\texttt{#1}}

% \raggedbottom

\lstnewenvironment{code}[1][]
  {\noindent
   \vspace{-0.5\baselineskip}
   \lstset{basicstyle=\ttfamily,
           frame=single,
           language=Haskell,
           keywordstyle=\color{black},
           #1}
   \fontsize{8pt}{8pt}\selectfont}
  {}

\newcommand\NOTE[1]{} % \mbox{}\marginpar{\fontsize{8pt}{8pt}\selectfont\raggedright\hspace{0pt}\emph{#1}}}

\def\maindocument{} % To tell tikz images that they are not stand alone

\begin{document}

\title{Integrating Theory Exploration in a Proof Assistant}

\author{Moa Johansson \and Dan Ros\'en \and Nicholas Smallbone}
\institute{Department of Computer Science and Engineering, Chalmers University of Technology
	\email{\{moa.johansson,nicsma,danr\}@chalmers.se}
	}

\authorrunning{Johansson, Ros\'en and Smallbone}
\titlerunning{Integrating Theory Exploration in a Proof Assistant}

\maketitle

\begin{abstract}
Theory exploration is a technique for automatically discovering novel and interesting lemmas in a given mathematical theory development. This paper describes Hipster, a system integrating theory exploration with the proof assistant Isabelle.
Existing Isabelle-tools like Sledgehammer can automatically select existing lemmas from the libraries that are likely to be needed for automation of many proofs. Hipster is an extension of this approach which also allows for discovery of new or missing lemmas which require induction, thus increasing the number of proofs that can be automated.
Hipster can be used in two main ways, either in exploratory mode to automatically generate basic lemmas about a given set of datatypes and functions in a new theory development, or in proof mode, for a particular proof attempt, trying to discover the missing lemmas which would allow the current goal to be proved. 

\end{abstract}

\section{Introduction}
\label{sec:intro}
The concept of theory exploration was first introduced by Buchberger \cite{buchberger2000theory}. He argues that as a contrast to  automated theorem provers that focus on proving one theorem at the time in isolation, mathematicians instead typically proceed by exploring entire theories, by conjecturing and proving layers of increasingly complex propositions. For each layer, appropriate proof methods are identified, and previously proved lemmas may be used to prove later conjectures. When a new concept (e.g. a new function) is introduced, we should prove a set of new conjectures which, ideally, "completely" relates the new with the old, after which other propositions in this layer can be proved easily by easy "routine" reasoning. Mathematical software should be designed to support this workflow. This is arguably what many interactive proof assistants, such as Theorema \cite{theorema} and Isabelle \cite{isabelle}, roughly support. However, they leave the generation of new conjectures relating different concepts largely to the user. Recently, a number of different systems have been implemented to address also this aspect of theory exploration \cite{McCasland2006,isacosy,isascheme,hipspecCADE}.  In this work, we take this one step further by integrating the discovery and proof of new conjectures in the workflow of the interactive theorem prover Isabelle. Our system, called Hipster, is based on our previous work on HipSpec \cite{hipspecCADE}, a theory exploration system for Haskell programs. In \cite{hipspecCADE} we showed that HipSpec is able to automatically discover many of the equational theorems present in Isabelle's list library. In this article we show how this can be used to speed up the development of new theories in Isabelle by discovering basic lemmas automatically. 
Isabelle theories are translated into Haskell, after which computation and evaluation in Haskell is used to efficiently generate generate interesting conjectures, which are then imported back into Isabelle and proved by automated tactics. Hipster can be used in two main ways: in \emph{exploratory mode} it quickly discovers basic properties about a newly defined function and its relationship to already existing ones. Hipster can also be used in \emph{proof mode}, to provide lemma hints for an ongoing proof attempt when the user is stuck. 

Our work also fits nicely with the increased popularity of Sledgehammer \cite{sledgehammer}, an Isabelle tool allowing the user to call various external automated provers. Sledgehammer use \emph{relevance filtering} to select among the available lemmas those likely to be useful for proving a given conjecture \cite{mash}. However, if a crucial lemma is missing, the proof attempt will fail. If theory exploration is employed before, we can increase the chances of Sledgehammer succeeding with little user effort. 

HERE: short example? Something unprovable by Sledgehammer before theory exploration.

%Hipster is parametrised by tactics for routine and "hard" reasoning. In this paper, we consider routine reasoning to be first-order resoning by Metis, give the conjectures Hipster has proved so far while "hard" reasoning is those proofs requiring induction. 

%Sledgehammer is a popular and powerful tool in Isabelle which allows users to call various external automated first-order provers \cite{sledgehammer}. Proofs are then replayed inside Isabelle's trusted kernel using its Metis prover. Sledgehammer's machine learning based relevance filter \cite{mash}, selects a number of facts from Isabelle's library that are judged to likely be relevant for the problem at hand. 
% 
%Which is because we can improve on Sledgehammer-style interactions in cases when the required lemmas simply are not available, so Sledgehammer would fail. 
Integrating the theory exploration system in an interactive setting has several other advantages, for instance when exploring a large theory which otherwise would generate a very large search space. The user can instead incrementally explore different relevant sub-theories while avoiding a search space explosion. Lemmas discovered in each sub-theory can be made available when exploring increasingly larger sets of functions. 

The article is organised as follows: In section \ref{sec:background} we give a brief overview of the HipSpec system which Hipster use to generate conjectures, after which we describe Hipster in more detail in section \ref{sec:hipster}. Blanchette variables in section X etc...


\section{Background}
\label{sec:background}
Short description of HipSpec and QuickSpec to go in here.

\section{Hipster}
This section shows a bunch of examples and explains how it all works.

% Dealing with Partial Functions
\section{Dealing With Partial Functions}
Isabelle is a logic of total functions. Nonetheless, we can define
apparently partial functions, such as \verb|hd|:
\begin{verbatim}
fun hd :: 'a list => 'a where
  hd (x#xs) = x
\end{verbatim}

How do we reconcile \verb|hd| being partial with Isabelle functions
being total? The answer is that in Isabelle, \verb|hd| is total, but
the behaviour of \verb|hd []| is unspecified: it returns some
arbitrary value of type \verb|'a|. Meanwhile in Haskell, \verb|head|
is partial, but the behaviour of \verb|head []| is specified: it
crashes. We must therefore translate \emph{partially defined} Isabelle
functions into \emph{total but underspecified} Haskell functions.

Hipster uses a technique suggested by Jasmin Blanchette
\cite{blanchettification} to deal with partial functions. Whenever we translate an Isabelle function
that is missing some cases, we need to add a default case, like so:
\begin{verbatim}
hd :: [a] -> a
hd (x:xs) = x
hd [] = ???
\end{verbatim}

But what should we put for the result of \verb|hd []|? To model the
notion that \verb|hd []| is unspecified, whenever we evaluate a test
case we will pick a \emph{random} value for \verb|hd []|. This value
will vary from test case to test case but will be consistent within
one run of a test case. The idea is that, if an equation involving
\verb|hd| in Haskell always holds, for all values we could pick for \verb|hd []|,
it will also hold in Isabelle, where the value of \verb|hd []| is unspecified.

Suppose we define the function \verb|second|, which returns the second
element of a list, as
\begin{verbatim}
second (x#y#xs) = y
\end{verbatim}
It might seem that we should translate \verb|second|, by analogy with \verb|hd|, as
\begin{verbatim}
second :: [a] -> a
second (x:y:xs) = y
second _ = ???
\end{verbatim}
and pick a random value of type \verb|a| to use in the default case.
But this translation is wrong! If we apply our translated \verb|second|
to a single-element list, it will give the same answer regardless of which
element is in the list, and HipSpec will discover the lemma
\verb|second [x] = second [y]|. This lemma is certainly not true of our
Isabelle function, which says nothing about the behaviour
of \verb|second| on single-element lists, and Hipster will fail to
prove it.

We must allow the default case to produce a different result for
different arguments. We therefore translate \verb|second| as
\begin{verbatim}
second :: [a] -> a
second (x:y:xs) = y
second xs = ??? xs
\end{verbatim}
where \verb|???| is a random \emph{function} of type \verb|[a] -> a|.
(QuickCheck can generate random functions.) As before, whenever we
evaluate a test case, we instantiate \verb|???| with a new random
function\footnote{To avoid having to retranslate the Isabelle theory
every time we evaluate a test case, in reality we parametrise the
generated program on the various \texttt{???} functions. That way,
whenever we evaluate a test case, we can cheaply change the default cases.}.
This second translation mimics Isabelle's semantics: any equation that
holds in Haskell no matter how we instantiate the \verb|???| functions
also holds in Isabelle.

In Hipster, we first use Isabelle/HOL's code generator to translate the
theory to Haskell. Then we transform \emph{every} function definition, whether it is
partial or not, in the same way we transformed \verb|second| above.
If a function is already total, the added case will
simply be unreachable. This avoids having to check functions for partiality.
The extra clutter introduced for total functions is not a problem as we neither reason about nor show the user the generated program.
  


\section{Use Cases}
Some form of evaluation or merge with Hipster section????

\section{Related Work}
%\section{Related Work}
\label{sec:related}


Hipster is an extension to our previous work on the HipSpec system \cite{hipspecCADE} designed for use in an interactive setting. HipSpec applies structural induction to conjectures generated by QuickSpec and send off proof obligations to external first-order provers. Hipster short-circuits this and directly import the conjectures back into Isabelle, allowing for more flexibility in the choice of tactic employed by letting the user control what is to be considered routine and difficult reasoning. Being inside Isabelle/HOL also provides the possibility to easily record lemmas for future use, perhaps in other theory developments,  the possibility to re-check proofs if required, as well as increased reliability as proofs have been run through Isabelle's trusted kernel. As Hipster use HipSpec for conjecture generation, any difference in performance (e.g. speed, lemmas proved) will depend only on what prover backend is used by HipSpec and what tactic is used by Hipster. %We remark that some tactics, such as Isabelle's simplifier, can sometimes even make Hipster faster than HipSpec despite running the proofs through Isabelle's , but this depend very much on the theory

There are two other theory exploration systems available for Isabelle/HOL, IsaCoSy \cite{isacosy} and IsaScheme \cite{isascheme}. They differ in the way they generate conjectures, and are both similar to Hipster in the kind of lemmas they can discover. A comparison between the system can be found in \cite{hipspecCADE}, showing that all three systems manage to find largely the same lemmas on theories about lists and natural numbers. HipSpec does however outperform the other two systems on speed.
IsaCoSy ensures that terms generated are non-trivial to prove by only generating irreducible terms, i.e. conjectures that do not have simple proofs by equational reasoning. These are then filtered through a counter-example checker and passed to IsaPlanner for proof \cite{isaplanner}. IsaScheme, as the name suggests, follow the scheme based approach first introduced for algorithm synthesis in Theorema \cite{theorema}. IsaScheme use general user specified schemas describing the shape of conjectures and then instantiate them with available functions and constants. It combines this with counter-example checking and Knuth-Bendix completion techniques in an attempt to produce a minimal set of lemmas. 
The counter-example checking in IsaCoSy and IsaScheme can however often be too slow for use in an interactive setting. As opposed to IsaCoSy, Hipster may generate reducible terms, but thanks to the congruence closure reasoning in QuickSpec, testing is much more efficient, and conjectures with trivial proofs are instead quickly filtered out at the proof stage. 
Neither IsaCoSy or IsaScheme has been used to generate lemmas in stuck proof-attempts, but only in fully automated exploratory mode. 

\emph{Proof planning critics} has been employed to analyse failed proof attempts in automatic inductive proofs \cite{productiveuse}. The critics use information from the failure in order to try to speculate a missing lemma top-down, using techniques based on rippling and higher-order unification. Hipster (and HipSpec) takes a less restricted approach, instead constructing lemmas bottom-up, from symbols available. As was shown in our previous work \cite{hipspecCADE}, this succeeds in finding lemmas that the top-down critics based approach fails to find, at the cost of possibly also finding a few extra lemmas as we saw in the example in section \ref{sec:comp-ex}.   

%Hipster has instead been designed specifically to be used interactively during theory development, allowing for more flexibility and user control, both in exploratory and proof mode. For instance, the user can specify what is to be considered difficult or routine reasoning, thus affecting the outcome of the exploration.  

%MATHsAiD, HR?
%Other systems not integrated in the workflow of a proof-assistant. With all the advantages that entail, recording discovered lemmas etc. Building up theories incrementally.

%Unlike HipSpec User can control search space by selecting which functions to pass in together. May run several passes of exploration with different combinations of functions. Parametrised by routine/difficult tactics for flexibility.


Other systems not integrated in the workflow of a proof-assistant. 


\section{Conclusion and Further Work}
%\section{Conclusion and Further Work}
\label{sec:concl}

Hipster integrates lemma discovery by theory exploration in the proof assistant Isabelle. We demonstrated two typical use cases of how this can help and speed up theory development: by generating interesting basic lemmas in \emph{exploration mode} or as a lemma suggestion mechanism for a stuck proof attempt in \emph{proof mode}. Hipster thus becomes a complement to tools like Sledgehammer, by discovery of missing lemmas, more proofs can be tackled automatically. 

The lemma discovery is currently limited to equational lemmas. We plan to extend the term generation and evaluation on the Haskell side to also take side conditions into account. For example, if theory exploration is called in the middle of a proof attempt, there may be assumptions associated with the current sub-goal, which should be taken into account when evaluating terms and dividing them into equivalence classes. 


%Conditionals, analogy between different datastructures when similar lemmas has been found about them. 

Conditionals, analogy between different datastructures when similar lemmas has been found about them. 

\bibliographystyle{plain}
\bibliography{bibfile}

\end{document}
