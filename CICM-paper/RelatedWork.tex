\section{Related Work}
\label{sec:related}


Hipster is an extension to our previous work on the HipSpec system \cite{hipspecCADE}. HipSpec applies structural induction to conjectures generated by QuickSpec and send off proof obligations to external first-order provers. Hipster short-circuits this and directly import the conjectures back into Isabelle, allowing for more flexibility in the choice of tactic employed by letting the user control what is to be considered routine and difficult reasoning. Being inside Isabelle also provides the possibility to easily record lemmas for future use, perhaps in other theory developments,  the possibility to re-check proofs if required, and increased trust as proofs have been run through Isabelle's trusted kernel. 

There are two other theory exploration systems available for Isabelle, IsaCoSy \cite{isacosy} and IsaScheme \cite{isascheme}. They differ in the way they generate terms, and are both similar to Hipster in the kind of lemmas they can discover. 
IsaCoSy ensures that terms generated are non-trivial to prove by only generating irreducible terms, i.e. conjectures that do not have simple proofs by equational reasoning. These are then filtered through a counter-example checker and passed to IsaPlanner for proof \cite{isaplanner}. The counter-example checking in IsaCoSy is however often too slow for use in an interactive setting. Hipster may generate reducible terms, but thanks to the congruence closure reasoning in QuickSpec, testing is much more efficient, and conjectures with trivial proofs are instead quickly filtered out at the proof stage. 
IsaScheme, as the name suggests, follow the scheme based approach first introduced for algorithm synthesis in Theorema \cite{theorema}. IsaScheme use general user specified schemas describing the shape of conjectures and then instantiate them with available functions and constants. It combines this with counter-example checking and Knuth-Bendix completion techniques in an attempt to produce a minimal set of lemmas. Neither IsaCoSy or IsaScheme has been used to generate lemmas for stuck proof-attempts, but only in fully automated exploratory mode. Hipster has instead been designed specifically to be used interactively during theory development, allowing for more flexibility and user control, both in exploratory and proof mode. For instance, the user can specify what is to be considered difficult or routine reasoning, thus affecting the outcome of the exploration.  

MATHsAiD, HR?
%Other systems not integrated in the workflow of a proof-assistant. With all the advantages that entail, recording discovered lemmas etc. Building up theories incrementally.

%Unlike HipSpec User can control search space by selecting which functions to pass in together. May run several passes of exploration with different combinations of functions. Parametrised by routine/difficult tactics for flexibility.

