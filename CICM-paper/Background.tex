\section{Background}
\label{sec:background}

In this section we give a brief overview of the HipSpec system which
we use as a backend for generating conjectures. HipSpec is a
state-of-the-art inductive theorem prover and theory exploration
system for Haskell. In \cite{hipspecCADE} we showed that HipSpec is
able to discover and prove the kind of equational lemmas present in
Isabelle's list library, when given list functions written in Haskell.

HipSpec works by first generating a set of conjectures about the
functions at hand. Its power comes from its capability to
automatically discover and prove lemmas, which are then used to help
subsequent proofs. In Hipster we are not interested in HipSpec's
proving capabilities: we use Isabelle for the proofs instead. Isabelle
is based on a small core of trusted axioms, and proofs must be built
on top of those axioms. In other words, we would have to reconstruct
inside Isabelle any proof we found anyway, so it is easier to use
Isabelle for the proofs in the first place. The part of HipSpec we use
is its conjecture synthesis system QuickSpec \cite{quickspec},
which efficiently generates equations about given functions and
datatypes.



QuickSpec takes a set of functions and constants as input, and proceeds to generate all type-correct terms up to a given limit (usually up to depth three). The generated terms are then divided into equivalence classes using the QuickCheck testing framework \cite{quickcheck}. QuickCheck generate values (something about generators...) for the variables occurring in the terms and evaluate the resulting ground terms.  Initially, all terms are in the same equivalence class, but after testing and evaluation, those that evaluate to different values are separated. This process is repeated until the equivalence classes stabilise, and resulting equations can be read off from each equivalence class. This means that the conjectures generated are, although not yet proved, fairly likely to be true.

Example, if we have space? 
