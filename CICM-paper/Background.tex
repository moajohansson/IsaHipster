\section{Background}
\label{sec:background}

In this section we give a brief overview of the HipSpec system which
we use as a backend for generating conjectures. HipSpec is a
state-of-the-art inductive theorem prover and theory exploration
system for Haskell. In \cite{hipspecCADE} we showed that HipSpec is
able to automatically discover and prove the kind of equational lemmas present in
Isabelle/HOL's libraries, when given the corresponding functions written in Haskell.

HipSpec works in two stages:
\begin{enumerate}
\item Generate a set of conjectures about the functions at hand. These
  conjectures are equations between terms involving the given
  functions, and have not yet been proved correct but are nevertheless
  extensively tested.

\item Attempt to prove each of the conjectures, using already proven conjectures as assumptions. HipSpec implements this by enumerating induction schemas, and firing off many proof obligations to automated first-order logic theorem provers.
\end{enumerate}
The proving power of HipSpec comes from its capability to
automatically discover and prove lemmas, which are then used to help
subsequent proofs.


In Hipster we can not directly use HipSpec's
proof capabilities (stage (2) above); we use Isabelle/HOL for the proofs instead. Isabelle  is an LCF-style prover which means that it
is based on a small core of trusted axioms, and proofs must be built
on top of those axioms. In other words, we would have to reconstruct
inside Isabelle/HOL any proof that HipSpec found, so it is easier
to use Isabelle/HOL for the proofs in the first place. 

The part of HipSpec we directly use
is its conjecture synthesis system (stage (1) above), called QuickSpec \cite{quickspec}),
which efficiently generates equations about a given set of functions and
datatypes.

QuickSpec takes a set of functions as input, and proceeds to generate all
type-correct terms up to a given limit (usually up to depth three). 
The terms may contain variables (usually at most three per type). 
These parameters are set heuristically, and can be modified by the user. 
QuickSpec attempts to divide the terms into equivalence classes such
that two terms end up in the same equivalence class if they are equal.
It first assumes that all terms of the same
type are equivalent, and initially puts them in the same equivalence class. 
It then picks random ground values for the variables in the terms
(using QuickCheck \cite{quickcheck}) and evaluates the terms.
If two terms in the same equivalence class evaluate to different
ground values, they cannot be equal; QuickSpec thus breaks each equivalence
class into new, smaller equivalence classes depending on what values
their terms evaluated to. This process is repeated until the
equivalence classes stabilise. We then read off equations from each
equivalence class, by picking one term of that class as a
representative and equating all the other terms to that representative.
This means that the conjectures
generated are, although not yet proved, fairly likely to be true, as they have been tested on several hundred different random values. The confidence increases with the number of tests, which can be set by the user. The default setting is to first run 200 tests, after which the process stops if the equivalence classes appear to have stabilised, i.e. if nothing has changed during the last 100 tests. Otherwise, the number of tests are doubled until stable.

As an example, we ask QuickSpec to explore the theory with list append,
\verb~@~, the empty list, \verb~[]~, and three list variables \verb~xs~,
\verb~ys~, \verb~zs~. Among the terms it will generate are:
\verb~(xs @ ys) @ zs~, \verb~xs @ (ys @ zs)~, \verb~xs @ []~ and \verb~xs~.
Initially, all four will be assumed to be in the same equivalence class.
The random value generator for lists from QuickCheck might for instance generate the values: \texttt{xs $\mapsto$ []}, \texttt{ys $\mapsto$ [a]} and \texttt{zs $\mapsto$ [b]}, where \texttt{a} and \texttt{b} are arbitrary distinct constants. Performing the substitutions of the variables in the four terms above and evaluating the resulting ground expressions gives us:

\begin{tabularx}{\textwidth}{l  X  X  X}
\\
 & Term & Ground Instance & Value \\
 \hline
1 \quad &\texttt{(xs @ ys) @ zs} & \texttt{([] @ [a]) @ [b]} & \texttt{[a,b]} \\
2 \quad&\texttt{xs @ (ys @ zs)} &\texttt{[] @ ([a] @ [b])} & \texttt{[a,b]}\\
3 \quad&\texttt{xs @ []} & \texttt{[] @ []} & \texttt{[]} \\
4 \quad &\texttt{xs} &\texttt{[]} & \texttt{[]} \\
\end{tabularx}
Terms 1,2  and 3,4 evaluate to the same value. The initial equivalence class will therefore be split in two accordingly.
After this, whatever variable assignments QuickSpec generates, the
terms in each class will evaluate to the same value. Eventually, QuickSpec stops and the equations for
associativity and right identity can be extracted from the resulting equivalence classes.

