\label{sec:evaluation}

In this section we present an evaluation of Hispter's scope of proving capability, an analysis which had not been performed for Hipster to the same extent priorly.

Keeping in mind evaluation of automated tools for interactive theorem proving necessarily has to consider some degree of interaction, two forms of evaluation have been carried out:

\begin{itemize}
\item case studies on algebraic data types and operations on them; in particular focusing on inductive theories for natural numbers and lists, of which we present a case study for insertion sort's correctness

\item evaluation on a subset of TIP \cite{tip-benchmarks}, a set of benchmarks and challenge problems for inductive theorem provers.
\end{itemize}

\subsection{Interactive case study: insertion sort}

We here showcase Hipster's handling of conditional lemmas via the proof of correctness for the theorem \isaCode{sorted (isort ts)}.
%
For it, we assume the less-or-equal operator \isaCode{$\le$} for naturals (and no prior, additional lemmas), the already defined \isaCode{sorted} and the function definitions:

\begin{lstlisting}[ mathescape, columns=fullflexible,keepspaces, basicstyle=\fontfamily{lmvtt}\selectfont, ]
fun insert :: "Nat $\Rightarrow$ Nat List $\Rightarrow$ Nat List" where
  "insert r []          = r # []"
| "insert r (t # ts) = (if (r $\le$ t) then r # (t # ts)	
                                           else t # (insert r ts))"

fun isort :: "Nat List $\Rightarrow$ Nat List" where
  "isort []          = []"
| "isort (t # ts) = insert t (isort ts)"
\end{lstlisting}

Running exploration from the simpler components is the first step, considering both equational and conditional lemmas, since we have two predicates involved in the definiens of functions in the final theorem.
%
Starting with conditional exploration for \isaCode{$\le$} and also for its negation, $10$ lemmas are yielded and proven, $6$ of which are conditionals (we present the vital lemmas towards the final proof) and all require recursion induction:

\begin{lstlisting}[ mathescape, columns=fullflexible,keepspaces, basicstyle=\fontfamily{lmvtt}\selectfont, ]
hipster_cond le
(...)
lemma lemma_ac [thy_expl]: "x $\le$ y $\Longrightarrow$ x $\le$ (S y) = True"
by (hipster_induct_schemes le.simps Nat.exhaust)

lemma lemma_ad [thy_expl]: "y $\le$ x $\Longrightarrow$ (S x) $\le$ y = False"
by (hipster_induct_schemes le.simps Nat.exhaust)
(...)
lemma lemma_ai [thy_expl]: "($\lnot$ (x $\le$ y)) $\Longrightarrow$ x $\le$ Z = False"
by (hipster_induct_schemes le.simps Nat.exhaust)
(...)
\end{lstlisting}

A new exploration considering the functions about sorting itself and taking \isaCode{sorted} as a side-condition for which to discover lemmas, Hipster discovers and proves both the conditional auxiliary lemma required and the goal theorem.

\begin{lstlisting}[ mathescape, columns=fullflexible,keepspaces, basicstyle=\fontfamily{lmvtt}\selectfont, ]
hispter_cond sorted isort insert
(...)
lemma isortInvariant [thy_expl]: "sorted ys $\Longrightarrow$ sorted (insert x ys) = True"
by (hipster_induct_schemes sorted.simps isort.simps insert.simps)
(...)
theorem isortSorts [thy_expl]: "sorted (isort x) = True"
by (hipster_induct_schemes sorted.simps isort.simps insert.simps)
\end{lstlisting}

During this last exploration, other interesting lemmas are discovered, all of which can be now proven automatically by using the sub-lemma about \isaCode{insert}'s invariant \isaCode{isortInvariant}:

\begin{lstlisting}[ mathescape, columns=fullflexible,keepspaces, basicstyle=\fontfamily{lmvtt}\selectfont, ]
lemma isortFixes [thy_expl]: "sorted x $\Longrightarrow$ isort x = x"
by (hipster_induct_schemes sorted.simps isort.simps insert.simps)

lemma insertComm [thy_expl]: "insert x (insert y z) = insert y (insert x z)"
by (hipster_induct_schemes insert.simps)
\end{lstlisting}

Invoking the recursion induction tactic \isaCode{hipster\_induct\_schemes} once proves all of the statements above, simplifying the interaction with the proof assistant; recursion induction therefore proves to be successful in automating proofs.
%
Particularly, the crucial lemma \isaCode{isortInvariant} is proven applying \isaCode{sorted}'s associated induction scheme, highlighting once again the need for support of conditional lemmas in automated inductive proving and the possibilities recursion induction brings.

% XXX: include? IF we mention new options
% Lastly, the benefit of an interactive environment is being able to parameterise the proving methods with adequate options as one works along. This supervised automation can be of great utility to a user during a formal theory development.


\subsection{Benchmarks: TIP}

From TIP, we evaluate Hipster over two subsets of problems employed in previous works on inductive theorem proving: Johansson, Dixon and Bundy's work on case-analysis for rippling \cite{IsaPcase} (we denote it \emph{case-analysis}), and prior work by Ireland and Bundy on employing proof failure to guide lemma discovery and patch inductive proofs \cite{productiveuse} (we denote it \emph{prod-failure}).

\subsubsection{Method}

To evaluate performance on TIP, each problem is analysed individually, in isolation from others.
%
Theory explorations were only run for those whenever the problem was not provable by the recursion induction tactic yet, i.e.: was missing helping lemmas.
%
These were first run on the individual functions appearing in the problem definition, jointly with their auxiliary functions, followed by explorations on groups of said functions if required, leaving conditional exploration as the last exploration to be run before defining the problem as non-provable by Hipster.

As already specified, conditional lemma discovery is limited to explore a single predicate to define the premise at a time. For the present evaluation this has sufficed.

Additionally, to test Hipster's capacity when working on strictly newly defined theories, no assumptions nor properties from theories in Isabelle/HOL were considered during proof search.
%
As an example, natural numbers are not Isabelle/HOL's, but redefined.
%
Hence, predefined notions of orderings and other properties do not play a part in proofs obscuring the results of Hipster's actual work.
%
In this way, we only consider as the base starting point a set of definitional statements, aligning with the purpose of proving based on structure and construction of programs.

\subsubsection{Results}

The following table summarises statistics on the two sets of the benchmarks run.
%
Columns \emph{EQ} and \emph{COND} do so for problems defined by an equational and a conditional theorem respectively.

\begin{tabularx}{\textwidth}{l | X X | X X}
  & \multicolumn{2}{c|}{Case-analysis} & \multicolumn{2}{c}{Prod-failure} \\
  &  EQ & COND & EQ & COND \\
  \hline
  Total & 71 & 15 & 38 & 12 \\
  \hline
  Fully automated & 67 & 14 & 29 & 12 \\
  Partially automated & 4 & 0 & 6 & 0 \\
  \hline
  Prior exploration & 33 & 4 & 34 & 4 \\
  Conditional exploration & 6 & 2 & 2 & 3 \\
  \hline
  Recursion induction & 29 & 6 & 1 & 1 \\
  Recursion induction on sub-lemmas & 16 & 1 & 5 & 0 \\
  \hline
  Filter theorems for FO prover & 3 & 0 & 0 & 0 \\ % XXX: remove this since we don't mention these options?
\end{tabularx}

\paragraph{Automation}
%
Overall, the rate of fully automated provability on the benchmark set is 90\%, considering fully automation the solving of a problem only with discovered lemmas and Hipster's automated recursion induction.
%
For conditional lemmas in the test set this rate is higher, 96\%.

A number of theorems required one of the following two similar lemmas, not automatically proven in a first instance:

\begin{lstlisting}[ mathescape, columns=fullflexible,keepspaces, basicstyle=\fontfamily{lmvtt}\selectfont, ]
len (x @ y) = len (y @ x)
count x (x @ y) = count x (y @ x)
\end{lstlisting}

% The failure is associated to smaller lemmas not being discovered, possibly because discovery of lemmas limits constructed terms to a certain depth.
\noindent These examples seem to indicate that recursion induction may not suffice when a non-commutative operation nested within another has commuting arguments at each side of an equality.
%
At least not in the absence of smaller related lemmas corresponding to subgoals.
%
This seems reasonable: the structure of the terms at each side of the equality will differ upon induction.

Nonetheless, proofs for the problems associated to these lemmas can be partially automated.
%
When extracted, the open subgoals are provable by Hipster.
%
Alternatively, for both cases, one can observe that the outermost function applied, \isaCode{len} and \isaCode{count} respectively, acts as a relator function between two datatypes.
%
Furthermore, these will in fact act as relator functions between list concatenation \isaCode{@} and addition for natural numbers \isaCode{plus}.
%
Adding interactively \isaCode{plus} to the exploration succeeds at discovering necessary lemmas for automating these two pending proofs without further intervention.

% TODO: successive inductions!

\paragraph{Theory exploration}

Just over half of the problems required prior lemma discovery, indicating the benefit of theory exploration.
%
Not surprisingly, equational lemmas were much more likely than conditionals to need exploration.
%
Equations are unrestricted and more general, hence potentially requiring more steps to asses their different case branches.
%
On the other hand, for these benchmarks, conditionals provided constraints that narrow down cases.

A smaller subset of problems were provable with the aid of conditional exploration, namely those involving functions defined in terms of some predicate.

\paragraph{Recursion induction}

Whereas recursion induction was not necessary as often as theory exploration (whether for the main theorem or auxiliary lemmas), its impact is still notable.
%
Structural induction was always attempted before by the tactic, meaning theorems solved via recursion induction (around a third of the benchmarks) would have not been solved otherwise, at least not with the degree of exploration carried out.
 
The results suggest recursion induction can save on exploration time.
%
It provides appropriate induction patterns that avoid the need for sub-lemmas about specific constructor combinations.


\subsubsection{Comparison}

% Briefly describe others tested against

\begin{tabularx}{\textwidth}{l | X X X | X X | X | X | X | X X X}
  & \multicolumn{3}{|c|}{Hipster}
  	& \multicolumn{2}{c|}{HipSpec}
  	& \multicolumn{1}{c|}{Zeno}
  	& \multicolumn{1}{c|}{IsaPlanner}
  	& \multicolumn{1}{c|}{CVC4}
  	& \multicolumn{3}{c}{Pirate} \\
  \hline
  Case-analysis & 80 & (82) & [83] & 80 & & 82 & 47 & 80 & 85 & & [86] \\
  Prod-failure & 41 & (47) & & 44 & (47) & 21 & - & 40 & & (47) & \\
\end{tabularx}

% Specify meaning of the numbers

% Draw conclusions and compare the different approaches; mainly with HipSpec and Pirate (as an example from which we can draw a difference with using generalisation)

% Emphasise interaction aspect


% \subsection{Discussion}   % Observed behaviour
