\label{sec:related}

The work on lemma discovery for inductive proofs has mainly focused on equational lemmas, for instance in the theory exploration systems IsaScheme and IsaCoSy \cite{isascheme,isacosy}, which also work on Isabelle/HOL theories. IsaScheme requires the user to provide \emph{term schemas}, which are then automatically filled in with available symbols. IsaCoSy only generates irreducible terms, and uses an internal constraint language to avoid generating anything that could be reduced by a known equation. These systems focused more on automation, while Hipster is designed to be useable in an interactive theory development. Hipster is faster, and now also supports conditional theory exploration where the user specifies an interesting condition. Conditional lemma discovery has also been missing from the IsaPlanner system, which uses \emph{proof critics} to deduce lemmas from failed proof attempts \cite{isaplanner2,IsaPcase}. 

Theory exploration systems rely on having an automated prover at hand to prove generated conjectures. In the context of inductive theories, most other automated provers supporting induction such as IsaPlanner, Zeno, HipSpec, Dafny and CVC4 \cite{isaplanner2, zeno, hipspecCADE,dafny,cvc4} only support structural induction. Hipster now also provides an automated tactic for recursion induction by exploiting Isabelle's automated derivation of such induction schemes. It can both be used in theory exploration and as a stand-alone automated tactic.

The use of recursion induction and the fact that Hipster produces LCF-style re-checkable proofs is also the main difference between Hipster and its sister system HipSpec \cite{hipspecCADE}, with which Hipster shares its conjecture generation component. HipSpec does instead rely on external first-order provers to solve the proof obligations arising in the step- and base-cases for inductive proofs, and does not produce checkable proofs.