\label{sec:conclusion}

Generation of conditional lemmas in theory exploration is a challenging problem, not least as it is difficult for a tool to automatically assess which side conditions are interesting.
%
Hipster is an interactive theory exploration system, and gets around this obstacle by relying on the user to decide which predicates are deemed interesting as conditions.
%
In this paper we have also presented a new automated tactic for recursion induction, which improves the level of proof automation of discovered conjectures in Hipster.
%
It can also be used as a powerful stand-alone induction tactic in Isabelle. 
%
Further work on the proving side includes experimenting with different heuristics for choosing which function's recursion induction scheme is most likely to produce a proof, as well as extending Hipster with tactics that can handle mutual- and co-induction automatically.
%Identifying which functions to pick for exploration, also from other theories (e.g. . plus and length)

%More efficient conjecture generation forthcoming in a soon to be released version of QuickSpec.
Hipster has various configuration options for adjusting which of the discovered lemmas are passed to its tactics in subsequent proofs.
%
For example, in larger theories, with many explorations, we may not want to pass all discovered lemmas to Isabelle's Metis tactic, as too many lemmas might slow down the proof process.
%
We plan to experiment with combining Hipster's tactics with the relevance filtering ideas used in Sledgehammer \cite{mash}.
%
Another item of further work is to extend Hipster to produce structured proofs in Isabelle's Isar language, instead of just a one-line application of Hipster's custom tactics.
%
This will be easier to read for a human user, and can be more streamlined, not needing to repeat the search done in the automatic proof found by Hipster's powerful tactics. 

